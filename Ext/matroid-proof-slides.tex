\documentclass[notheorems]{beamer}
\usepackage{latexsym}
\usepackage{amsmath,amssymb}
\usepackage{color,xcolor}
\usepackage{graphicx}
\usepackage{listings}
\usepackage{hyperref}
\usepackage{algorithm}
\usepackage{amsthm}

\usetheme{Madrid}
\usefonttheme[onlymath]{serif}
\usecolortheme{rose}

\setbeamertemplate{theorems}[numbered]

\newtheorem{theorem}{Theorem}
\newtheorem{lemma}[theorem]{Lemma}
\newtheorem{proposition}[theorem]{Proposition}
\newtheorem{corollary}[theorem]{Corollary}
\newtheorem{exercise}{Exercise}
\newtheorem*{solution}{Solution}
\newtheorem{definition}{Definition}

\begin{document}

\title[The Proof of Disjoint Path Matroid]{The Proof of Disjoint Path Matroid}
\author[Hongjie Fang]{Hongjie Fang}
\institute[CSE, SJTU] {Computer Science and Engineering Department, Shanghai Jiao Tong University}
\date[\today]{\today}
\frame{\titlepage}

\begin{frame}
\frametitle{Disjoint Path Matroid}
The definition of the Disjoint Path Matroid is as follows.
\begin{definition}[Disjoint Path Matroid]
Let $G = (V,E)$ be an arbitrary directed graph, and let $s$ be a fixed vertex of $G$. A subset $I \subseteq V$ is independent if and only if there are edge-disjoint paths from $s$ to each vertex in $I$. Let $\mathbf{C}$ be the collection of independent subsets of $V$, then $M = (V, \mathbf{C})$ is a Disjoint Path Matroid.
\end{definition}
And we want to prove that $M$ is indeed a matroid.
\end{frame}

\begin{frame}
\frametitle{Hereditary}
\begin{lemma}[Hereditary]
The collection $\mathbf{C}$ is hereditary.
\end{lemma}
\begin{proof}
$\forall B \in \mathbf{C}$, $B$ is independent, that is, there are edge-disjoint paths from $s$ to each vertex in $B$. $\forall A \subseteq B$, then there are edge-disjoint paths from $s$ to each vertex in $A$, since the vertex in $A$ must be in $B$. Hence, $A \in \mathbf{C}$.
Therefore, the collection $\mathbf{C}$ is hereditary.
\end{proof}
\end{frame}

\begin{frame}
\frametitle{Overlapped Path}
Assume that
\begin{itemize}
  \item All lemmas and definitions are under the condition of a given graph $G=(V,E)$;
  \item A path $P$ consists of the edges in path, so $P$ is a subset of $E$;
  \item $dst(P)$ stands for the destination of the path $P$.
\end{itemize}
\pause
Before we prove the exchange property of $M$, I would like to propose the following useful definitions and lemmas.
\begin{definition}[Overlapped Path]
Two paths $P_1, P_2$ is overlapped if and only if there exists an edge $e \in E$ satisfying $e \in P_1$ and $e \in P_2$ (i.e., $P_1 \cap P_2 \ne \varnothing$).
\end{definition}
\end{frame}

\begin{frame}
\frametitle{Construction Lemma}
\begin{lemma}[Construction Lemma]
Suppose $A \in \mathbf{C}, B \in \mathbf{C}$, and $m = |A| < |B| = n$. According to the definition of $\mathbf{C}$, there exist $m$ non-overlapped paths with the destinations of every vertex in $A$, say $\mathbf{P} = \{P_1, P_2, ..., P_m\}$. Similarly, there also exist $n$ non-overlapped paths with the destinations of every vertex in $B$, say $\mathbf{Q} = \{Q_1, Q_2, ..., Q_n\}$. Then we are able to construct a non-overlapped path set $\mathbf{R} = \{R_1, R_2, ..., R_n\}$ satisfying
\begin{displaymath}
A \subseteq \{dst(R_i) \mid i = 1,2,...,n\}
\end{displaymath}
\end{lemma}
\end{frame}

\begin{frame}
\frametitle{Proof of Construction Lemma}
\begin{itemize}
\item Initially, we set $\mathbf{R}_0 = \mathbf{Q}$. Then we prove that after $c$ steps ($c \in \mathbb{R}$), $\mathbf{R}_c$ meets the requirement $A \subseteq \{dst(R) \mid R \in \mathbf{R}_c\}$.
\pause
\item In every step, we will maintain the {\color{blue}property} that each path in $\mathbf{R}_i$ is not overlapped with other paths in $\mathbf{R}_i$.
\pause
\item First, we give the following process of every step, and try to prove that if the process ends, then it will give the correct answer to the problem.
\item In $i$-th step, we will check that whether $\mathbf{R}_i$ meets the requirement $A \subseteq \{dst(R) \mid R \in \mathbf{R}_i\}$ first.
\item If the answer is true, then let $c = i$, we prove that $\mathbf{R}_c$ meets the requirement above.
\item If the answer is false, then there exists $P \in \mathbf{P}$ satisfying $dst(P) \notin \{dst(R) \mid R \in \mathbf{R}_i\}$. We will discuss this case in next slide.
\end{itemize}
\end{frame}

\begin{frame}
\frametitle{Proof of Construction Lemma (Cont.)}
\begin{itemize}
\item We divided it into two cases, according to the situation of $P$.
\pause
\begin{itemize}
\item {\color{red} Case 1} ($P$ is not overlapped with any paths in $\mathbf{R}_i$):

Randomly choose a path in $\mathbf{R}_i$ called $R'$, which is not in $\mathbf{P}$ (there must exist because paths are distinct in $\mathbf{P}$ and $|\mathbf{P}| < |\mathbf{R}_i|$). Then $\mathbf{R}_{i+1} = \mathbf{R}_i - \{R'\} + \{P\}$ will maintain the {\color{blue}property} above, since every two paths in $\mathbf{R}_i$ do not have overlapped edge and $P$ is not overlapped with any paths in $\mathbf{R}_i$.
\pause
\item {\color{red} Case 2} ($P$ is overlapped with some paths in $\mathbf{R}_i$):

Suppose the last overlapped edge in $P$ is $e$, and it is the overlapped edge between $P$ and $R'$, where $R' \in \mathbf{R}_i$. Then we construct a new path $P'$ by mainly copying path $R'$, except that after edge $e$ it follows path $P$. Therefore, the destination of $P'$ is the same as the destination of $P$, and $P'$ do not overlapped with any path in $\mathbf{R}_i - \{R'\}$. It's because the front part of $P'$ follows $R'$, which is not overlapped with any other paths in $\mathbf{R}_i$, and the last part of $P'$ follows path $P$ after edge $e$, which is not overlapped with any paths since $e$ is the last overlapped edge in $P$. Hence $\mathbf{R}_{i+1} = \mathbf{R}_i - \{R'\} + \{P'\}$ will maintain the {\color{blue}property} above.
\end{itemize}
\end{itemize}
\end{frame}


\begin{frame}
\frametitle{Proof of Construction Lemma (Cont.)}
\begin{itemize}
\item In summary, we have proved that if the process ends successfully, it will give us the correct answer to the question.
\pause
\item Second, we prove that the process will end in finite steps.
\pause
\item Notice that when the first case happens, we substitutes an element in $\mathbf{R}$ with an element in $\mathbf{P}$ and keeps the elements in $\mathbf{R}$ distinct. And the second case won't change these elements in $\mathbf{P}$. Therefore, the first case happens at most $|\mathbf{P}| = m$ times.
\pause
\item Now let us consider the second case. Let $E_i$ be the \textbf{multi-set} (i.e. set that allows duplicate elements) of edges that have been operated after $i$-th second-case step. At first, we stipulate that $E_0 = \varnothing$. In $i$-th second-case step $(i\in\mathbb{N}^+)$, we set $e$ as a new operated edge, that is, $E_i = E_{i-1} + \{e\}$. Then it is obvious that $|E_i|$ is strictly increasing.
\end{itemize}
\end{frame}

\begin{frame}
\frametitle{Proof of Construction Lemma (Cont.)}
\begin{itemize}
\item Notice that an edge can not be operated twice, that is, the multi-set $E_i$ is actually a set. In the second case, we set edge $e$ as operated edge and we know that $e \in \cup{\mathbf{P}}$ because of overlapping. After the setting, edge $e$ cannot be set as operated edge again because:
    \begin{itemize}
    \item Edge $e$ cannot be overlapped again by another path in $\mathbf{P} - \{P\}$, and for path $P$ we won't operate the $e$ again because $dst(R') \in A$;
    \item If an edge $e'$ in path $R'$ before edge $e$ is operated in future, then $e$ will maintain the property that $e \in \cup\mathbf{P}$ and $e \in \cup\mathbf{Q}$, which will ensure that it won't be operated again.
    \end{itemize}
\pause
\item Since the process starts with an empty set $E_0 = \varnothing$ and only involves edges in a finite set $\mathbf{E} \stackrel{\Delta}{=} (\cup{\mathbf{P}}) \cup (\cup{\mathbf{Q}})$, the second case happens at most $|\mathbf{E}|$ times.
\item Therefore, the whole process will end in at most $(|\mathbf{E}| + m)$ steps. Thus, the lemma is proved. QED.
\end{itemize}
\end{frame}

\begin{frame}
\frametitle{Exchange Property}
\begin{lemma}[Exchange Property]
The independent system $M = (V, \mathbf{C})$ satisfies the exchange property.
\end{lemma}
\begin{proof}
$\forall A\in \mathbf{C}, \forall B\in \mathbf{C}$ and $|A| < |B|$, with the Construction Lemma, we can construct a path set $\mathbf{R}$ satisfying
\begin{displaymath}
A \subseteq \{dst(R) \mid R \in \mathbf{R}\}
\end{displaymath}
Thus, it's easy to choose $(|A| + 1)$ paths in $\mathbf{R}$ and keep the property above, and we suppose these paths form path set $\mathbf{R}'$. Then, we construct the set $C$ as follows.
\begin{displaymath}
C = \{dst(R) \mid R \in \mathbf{R}'\}
\end{displaymath}
It's easy to see that $A \subseteq C$ and $C \in \mathbf{C}$. Thus the exchange property is proved.
\end{proof}
\end{frame}

\begin{frame}
\frametitle{Final: Matroid}
\begin{theorem}
Disjoint Path Matroid $M$ is indeed a matroid.
\end{theorem}
\begin{proof}
According to the Lemma Hereditary and Lemma Exchange Property, the theorem is proved.
\end{proof}
\end{frame}

\end{document}
